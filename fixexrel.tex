\documentclass[11pt]{amsart}
\usepackage[margin=1in, marginparwidth=0.8in]{geometry}
\usepackage[colorlinks,
            linkcolor=black!50!red,
            citecolor=blue,
            pdfpagemode=None]{hyperref}
\usepackage{cleveref}

\usepackage[draft]{say}
\newcommand{\saySS}[1]{\say[SS]{#1}}

% shorthands 
\newcommand{\cA}{\mathcal{A}}
\newcommand{\bC}{\mathbb{C}}
\newcommand{\cAb}{\mathcal{A}_\bullet}
\newcommand{\Supp}{\operatorname{Supp}}
\newcommand{\ZZ}{\mathbb{Z}}
\newcommand{\bfg}{\mathbf{g}}
\newcommand{\bfc}{\mathbf{c}}

% ambients and numbering
\newtheorem{theorem}{Theorem}[section]
\newtheorem{conjecture}[theorem]{Conjecture}
\newtheorem{corollary}[theorem]{Corollary}
\newtheorem{definition}[theorem]{Definition}
\newtheorem{lemma}[theorem]{Lemma}
\newtheorem{proposition}[theorem]{Proposition}
\newtheorem{example}[theorem]{Example}
\newtheorem{remark}[theorem]{Remark}
\numberwithin{equation}{section}

\begin{document}
\title[Exchange relations for finite type]{Exchange relations for finitie type cluster algebras with acyclic initial seed and principal coefficients}

\author[Felikson]{Anna Felikson}
\address[Anna Felikson]{Durham University}
\email{email-here}

\author[Reading]{Nathan Reading}
\address[Nathan Reading]{North Carolina State Univesity}
\email{email-here}

\author[Stella]{Salvatore Stella}
\address[Salvatore Stella]{IN$d$AM - Marie Curie Actions fellow, Universit\`a\`{ }La Sapienza'', Roma, Italy.}
\email{stella@mat.uniroma1.it}

\author[tumarkin]{Pavel Tumarkin}
\address[Pavel Tumarkin]{Durham University}
\email{email-here}

\begin{abstract}
We give explicit description of all the exchange relations for finitie type cluster algebras with acyclic initial seed and principal coefficients.
\end{abstract}

\maketitle

\section{Introduction}
In \cite{Ste13} the third author gave a formula for all the exchange relations in any finite type acyclic cluster algebra without coefficients.
This formula required only to know the initial exchange matrix, or rather the \emph{Coxeter element} naturally associated to it, and the \emph{$\bfg$-vectors} of the two cluster variables involved.
Our goal here is to improve on this result to include principal coefficients.

Let $B$ be any finite type acyclic 

\begin{theorem}
  In any cluster algebra of finite type with principal coefficients at an acyclic initial seed the coefficient appearing in the exchange relation of any two cluster variables is \emph{explicitly} determined by their $\bfg$-vectors. 
\end{theorem}

The claim in previous statement lays in the word ``explicitly''. 
Indeed, since cluster variables are parametrized by their $\bfg$-vectors, and since exchange relations are prescribed by the cluster variables involved, it is obvious that the $\bfg$-vectors determines indirectly the coefficients involved. 
This paper is about giving precise formulas to compute them.

\begin{theorem}
  Let $\lambda$ and $\mu$ be exchangeable weights. 
  Then the exponent $\alpha$ in the exchange relation
  \[
    x_\lambda x_\mu = x_{\lambda+\mu} + y^\alpha x_{\lambda\uplus_c\mu}
  \]
  is the unique positive root satisfying both
  \[
    -B\alpha = \lambda+\mu-\lambda\uplus_c\mu
  \]
  and
  \[
    \lambda(\alpha^\vee)  \mu(\alpha^\vee) = -1.
  \]
\end{theorem}

\section{Proofs}

Recall well known fact
\begin{lemma}
  Let $B$ be any $n\times n$ acyclic matrix of finite type.
  If the type of $B$ is not $D_n$, then its kernel has dimension $0$ if $n$ is even and $1$ if $n$ is odd.
  It the type of $B$ is $D_n$, then its kernel has dimension $2$ if $n$ is even and $1$ if $n$ is odd.
\end{lemma}

Let $V$ be the $n$-dimensional real vector space containing the root system $\Phi$.
We always write vectors in $V$ in the basis of simple roots $\alpha_i$.
The \emph{support} of $v\in V$ is the subset $\Supp(v)\subseteq [1,n]$ consisting of all the indices of nonzero coordinates of $v$.
Such a subset induces a full subdiagram of the Dynkin diagram of $\Phi$; when we talk about the \emph{connected components} of $\Supp(v)$ we refer to the connected components of this subdiagram.

% bibliography
\bibliographystyle{amsalpha}
\bibliography{bibliography}

\end{document}

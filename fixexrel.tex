\documentclass[11pt]{amsart}
\usepackage[margin=1in, marginparwidth=0.8in]{geometry}
\usepackage[colorlinks,linkcolor=black!50!red,citecolor=blue,pdfpagemode=None]{hyperref}
\usepackage[capitalise]{cleveref}

\usepackage[draft]{say}
\newcommand{\saySS}[1]{\say[SS]{#1}}

% shorthands 
\newcommand{\cA}{\mathcal{A}}
\newcommand{\bC}{\mathbb{C}}
\newcommand{\cAb}{\mathcal{A}_\bullet}
\newcommand{\Supp}{\operatorname{Supp}}
\newcommand{\ZZ}{\mathbb{Z}}
\newcommand{\bg}{\mathbf{g}}
\newcommand{\bc}{\mathbf{c}}

% ambients and numbering
\newtheorem{theorem}{Theorem}[section]
\newtheorem{conjecture}[theorem]{Conjecture}
\newtheorem{corollary}[theorem]{Corollary}
\newtheorem{definition}[theorem]{Definition}
\newtheorem{lemma}[theorem]{Lemma}
\newtheorem{proposition}[theorem]{Proposition}
\newtheorem{example}[theorem]{Example}
\newtheorem{remark}[theorem]{Remark}
\numberwithin{equation}{section}

\begin{document}
\title[Exchange relations for finite type]{Exchange relations for finitie type cluster algebras with acyclic initial seed and principal coefficients}

\author[Felikson]{Anna Felikson}
\address[Anna Felikson]{Durham University}
\email{email-here}

\author[Reading]{Nathan Reading}
\address[Nathan Reading]{North Carolina State Univesity}
\email{email-here}

\author[Stella]{Salvatore Stella}
\address[Salvatore Stella]{IN$d$AM - Marie Curie Actions fellow, Universit\`a\`{ }La Sapienza'', Roma, Italy.}
\email{stella@mat.uniroma1.it}

\author[tumarkin]{Pavel Tumarkin}
\address[Pavel Tumarkin]{Durham University}
\email{email-here}

\begin{abstract}
We give an explicit description of all the exchange relations for finitie type cluster algebras with acyclic initial seed and principal coefficients.
\end{abstract}

\maketitle

\section{Introduction}
  In \cite{Ste13} the third author gave a formula for all the exchange relations in any finite type acyclic cluster algebra without coefficients.
  Our main theorem here improves on this result to include principal coefficients.
  
  \begin{theorem}
    \label{thm:main_vague}
    In any cluster algebra of finite type with principal coefficients at an acyclic initial seed, the coefficient appearing in the exchange relation of any two exchangeable cluster variables is \emph{explicitly} determined by their $\bg$-vectors. 
  \end{theorem}
  
  The claim in previous statement lays in the word ``explicitly''. 
  Indeed, since cluster variables are parametrized by their $\bg$-vectors, and since exchange relations are prescribed by the cluster variables involved, it is obvious that the $\bg$-vectors determines indirectly the coefficients. 
  This note is about giving precise formulas to compute them.
 
  In order to make \cref{thm:main_vague} more precise we need to recall few notions and results from \cite{Ste13,YZ08}.

  Let $A$ be any finite type \emph{Cartan matrix} and denote by $W=\langle s_1,\dots,s_n\rangle$ the associated \emph{Weyl group} and \emph{simple reflections}.
  To each \emph{Coxeter element} $c=s_{i_1}\cdots s_{i_n}$ in $W$ we can associate a skew-symmetrizable integer matrix $B_c=(b_{ij})_{i,j\in[1,n]}$ by setting
  \[
    b_{ij}=
    \begin{cases}
      -a_{ij} & \text{if } i\prec_c j  \\
      a_{ij}  & \text{if } j\prec_c i  \\
      0       & \text{otherwise}
    \end{cases}
  \]
  where we write $i\prec_c j$ if and only if $s_i$ precedes $s_j$ in all reduced expressions of $c$.
  As $c$ varies we get all the possible \emph{acyclic} exchange matrices in any cluster algebra of the same type as $A$.
  We will denote by $\cA(c)$ (resp. $\cA_0(c)$) the cluster algebra with initial exchange matrix $B_c$ and  \emph{principal} (resp. \emph{without}) \emph{coefficients} at the inital seed.

  Let $\omega_i$ be the i-th \emph{fundamental weight} in the \emph{weight lattice} $P$ of $A$; we will routinely identify $\ZZ^n$ with $P$ by means of the basis of fundamental weights.
  Let $w_0$ be the longest element of $W$ and denote by $h(i;c)$ the minimum positive integer such that 
  \[
    c^{h(i;c)}\omega_i = w_0\omega_i
  \]
  (it is a finite number \cite[Proposition 1.3]{YZ08}).
  \begin{theorem}[{\cite[Theorem 1.4]{YZ08}}]
    The cluster variables of $\cA(c)$ are naturally in bijection with the elements of the set
    \[
      \Pi(c)
      :=
      \left\{
        c^m\omega_i \, :\, i\in[1,n] \, , \, 0\leq m \leq h(i;c) 
      \right\}.
    \]
    To the cluster variable $x_\lambda$ corresponds its $\bg$-vector $\lambda\in\Pi(c)$.
  \end{theorem}
  This correspondence extends to a bijection between points of $P$ and \emph{cluster monomials} of $\cA(c)$ (cf. \cite[Theorem 1.2]{Ste13}); for $\lambda\in P$ we will denote by $x_\lambda$ the cluster monomial whose $\bg$-vector is $\lambda$.
  In view of the ``separation of additions formula'' we get a similar bijection with cluster monomials in $\cA_0(c)$; we will freely refer to either of the two according to our needs.

  The set $\Pi(c)$ is naturally endowed with a permutation $\tau_c$ defined by
  \[
    \tau_c (\lambda) 
    :=
    \begin{cases}
      \omega_i  & \text{if $\lambda = -\omega_i$} \\
      c\lambda  & \text{otherwise}
    \end{cases}
  \]
  which extends to a piecewise linear map on the whole of $P$.
  This is a combinatorial shadow of a notable automorphism of $\cA(c)$ sending $x_\lambda$ to $x_{\tau_c(\lambda)}$. 

  Let $Q$ be the \emph{root lattice} of $A$ with simple roots $\alpha_i$; as for $P$, we will identify $Q$ and $\ZZ^n$ using the basis of simple roots.
  \begin{definition}
    The \emph{compatibility degree} $(\bullet||\bullet)_c$ is the unique $\tau_c$-invariant function on pairs of elements of $\Pi(c)$ defined by the initial conditions
    \[
      (\omega_i||\lambda)_c
      :=
      \left[ (c^{-1}-1)\lambda ; \alpha_i\right]_+
    \]
    where $[v,\alpha_i]$ is the coefficient of $\alpha_i$ in $v$ expressed in the basis of simple roots and $[m]_+$  denotes $\max\{m, 0\}$ (cf. \cite[Proposition 5.1]{YZ08}).
  \end{definition}
  The name comes from the following important property.
  \begin{proposition}
    Two weights $\lambda$ and $\mu$ from $\Pi(c)$ are
    \begin{itemize}
      \item
        \emph{compatible} (i.e. there is a cluster of $\cA(c)$ containing both $x_\lambda$ and $x_\mu$) if and only if
        \[
          (\lambda||\mu)_c = 0
          \quad \quad
          \text{(equivalently $(\mu||\lambda)_c=0$)}
        \]

      \item
        \emph{exchangeable} (i.e. there are two clusters of $\cA(c)$ that can be obtained from one-another by swapping $x_\lambda$ for $x_\mu$) if and only if
        \[
          (\lambda||\mu)_c = 1 = (\mu||\lambda)_c
        \]
    \end{itemize}
  \end{proposition}

  Our starting point is the following restatement of \cite[Propositions 5.1 and 5.2]{Ste13}. 
  \begin{proposition}
    Suppose $\lambda$ and $\mu$ are exchangeable weights in $\Pi(c)$. 
    Then the set
    \[
      \left\{
        \tau_c^{-m}\left(\tau_c^m(\lambda)+\tau_c^m(\mu)\right)
      \right\}_{m\in\ZZ}
    \]
    consists precisely of two elements. One of them is $\lambda+\mu$; denote the other by $\lambda\uplus_c\mu$.

    The exchange relation of $x_\lambda$ and $x_\mu$ in $\cA_0(c)$ is then
    \[
      x_\lambda x_\mu = x_{\lambda+\mu} + x_{\lambda\uplus_c\mu}.
    \]
  \end{proposition}

  Denote by $y_1,\dots,y_n$ the generators of the coefficient semifield of $\cA(c)$ and by $y^\alpha$ the product $\prod_{i=1}^n y_i^{[\alpha;\alpha_i]}$.
  In view of \cite{NS14}, taking into account that the exchange relations in $\cA(c)$ are homogeneous, we get the following straigthforward corollary.
  \begin{corollary}
    The exchange relations in $\cA(c)$ are all of the form 
    \[
      x_\lambda x_\mu = x_{\lambda+\mu} + y^\alpha x_{\lambda\uplus_c\mu}
    \]
    for some exchangeable weights $\lambda$ and $\mu$ and some positive root $\alpha$.
  \end{corollary}

  We can finally recast \cref{thm:main_vague} into a more precise statement.
  \begin{theorem}
    \label{thm:main}
    Let $\lambda$ and $\mu$ be exchangeable weights. 
    Then the exponent $\alpha$ in the exchange relation
    \[
      x_\lambda x_\mu = x_{\lambda+\mu} + y^\alpha x_{\lambda\uplus_c\mu}
    \]
    is the unique positive root satisfying both
    \[
      -B_c\alpha = \lambda+\mu-\lambda\uplus_c\mu
    \]
    and
    \[
      \alpha^\vee(\lambda) \alpha^\vee(\mu) = -1.
    \]
  \end{theorem}
  
  \section{Proofs}
  
  Recall well known fact
  9\begin{lemma}
    Let $B$ be any $n\times n$ acyclic matrix of finite type.
    If the type of $B$ is not $D_n$, then its kernel has dimension $0$ if $n$ is even and $1$ if $n$ is odd.
    It the type of $B$ is $D_n$, then its kernel has dimension $2$ if $n$ is even and $1$ if $n$ is odd.
  \end{lemma}
  
  Let $V$ be the $n$-dimensional real vector space containing the root system $\Phi$.
  We always write vectors in $V$ in the basis of simple roots $\alpha_i$.
  The \emph{support} of $v\in V$ is the subset $\Supp(v)\subseteq [1,n]$ consisting of all the indices of nonzero coordinates of $v$.
  Such a subset induces a full subdiagram of the Dynkin diagram of $\Phi$; when we talk about the \emph{connected components} of $\Supp(v)$ we refer to the connected components of this subdiagram.

% bibliography
\bibliographystyle{amsalpha}
\bibliography{bibliography}

\end{document}
